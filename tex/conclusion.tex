\documentclass[../structure.tex]{subfiles}
\begin{document}
\chapter{Conclusion}
In this thesis, the non-rigid ICP registration method previously discussed in \cite{Amberg2007} was demonstrated, customized and implemented in chapter \textit{Methods and Implementation} to fit brain fiber tract data, which is considered a challenging registration problem.

The tools we have designed produce sufficient results when tested using real brain pathways data from patients from \textbf{\textit{UKB Universitätsklinikum Bonn}}. The data is images which were taken by MRI technology and converted into 3D coordinates.

The tool can handle the deformation by signing affine matrix for each vertex to optimally align overall pathway.

The performance of the tool can be improved by applying hierarchical clustering and apply soft membership to vertices depending on the central clusters points new coordinate.

Applying PCA sometimes does not find the right orientation of bundles due to similarity of vertices distribution and that gives similar principle components .

\end{document}