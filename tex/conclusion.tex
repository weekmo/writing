\documentclass[../structure.tex]{subfiles}
\begin{document}
\chapter{Conclusion}
In this thesis non-rigid ICP registration method that discussed in \cite{Amberg2007} was illustrated, customized and implemented in chapter \textit{Methods and Implementation} to fit the brain pathways data which considered as challenging registration problem.

The tool gives sufficient results as shown in chapter \textit{Results} when was tested using real brain pathways data from patients from \textbf{\textit{UKB Universitätsklinikum Bonn}}. The data is images were taken by MRI technology and converted to 3D coordinate.

The tool can handle the deformation by signing affine matrix for each vertex to optimally align overall pathway.

The performance of the tool can be improved by applying hierarchical clustering and apply soft membership to vertices depending on the central clusters points new coordinate.

Applying PCA sometimes does not find the right orientation of bundles due to similarity of vertices distribution and that gives similar principle components .

\end{document}