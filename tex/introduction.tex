\documentclass[../structure.tex]{subfiles}
%\usepackage{../mypkg}
\begin{document}
\chapter{Introduction}

The thesis is about customizing \textit{Non-rigid Iterative Closets Point Registration}, which discussed in \cite{Amberg2007}, to process 3D brain fiber pathways. The alignment of this type of data is not a trivial task as there is variability  in the shapes of the same fiber pathways in different brains and alignment between the left and right side of the fiber pathway can also be challenging. We illustrate the method in details in chapter \textit{Methods and Implementation} and we develop tools that can contribute of solving the difficulty of dealing with such data. The tools can assign local transformation which gives sufficient registration of fiber pathways. In chapter \textit{Result} we present result after testing our tools with real data from patients and controls which is collected at \textit{UKB Universitätsklinikum Bonn}. In the last chapter \textit{Conclusion} we give advice of how can we improve our tools and further work that can be done.

\end{document}


