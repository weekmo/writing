\documentclass[../structure.tex]{subfiles}
%\usepackage{../mypkg}
\begin{document}
\chapter{Introduction}

-complexity of brain
-Data set and usage of registration (all the details about the data) and the 3D model
-Importance of registration (differences case and control)
-motivation/Proposal of the thesis - comparison between non-rigid and rigid image registration 
-Medical application

\begin{itemize}
\item brain anatomy
\item how we take MRI
\item how we build 3D model
\item The difficulty of aligning pathways

\end{itemize}

As we mention before, registration is important step on many application including bioinformatics, which is our case, because our data is brain pathways (sometimes in this thesis we call the bundles or tracts for simplicity). Our method will help comparing passion and control brain pathways to distinguish the deformation occur on patients brains due to disease or outside affects. we apply ICP method refer to \cite{Amberg2007} with some altering to fit the spacial case of our data, because the method on \cite{Amberg2007} applied for surface registration and our data is tracts (streamlines).

Our data as we mentioned is brain pathways which are part of white matter (brain consists of white and gray matter as we will explain later), 
\end{document}


