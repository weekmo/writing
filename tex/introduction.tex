\documentclass[../structure.tex]{subfiles}
%\usepackage{../mypkg}
\begin{document}
\chapter{Introduction}
In order to compare the differences in the fiber pathways of cases and control to study various brain related disorders, this thesis focuses on customizing \textit{Non-rigid Iterative Closest Point Registration}, which is discussed in \cite{Amberg_2007}, to process 3D brain fiber pathways. Due to differences between pathways in different subjects , the alignment of this type of data is not a trivial task. Also, there is variability in the shapes of the same fiber pathways in different brains and alignment between the left and right side of the fiber pathway can also be challenging. We illustrate the method in detail in the chapter \textit{Methods and Implementation} where we develop tools that can contribute to solve the problem of dealing with such data. The tools can assign local transformations, which gives sufficient registration of fiber pathways. 

LSQR, a linear solver is the state of art method which is used to reduce the cost function or Euclidean distance iteratively in our case to reach the minimal overall distance with respect to stiffness which prevents the template graph to totally map to the target graph. An advantage of using LSQR is linearly solving the cost function which consists of matrices by assigning local transformations (local transformations mean affine matrix for each vertex in the template graph). But the method is slow and can be improved in performance by applying hierarchical clustering (already mentioned in the chapter \textit{conclusion}). Another method, DIPY transformation uses an optimizer to reach the optimal combination of affine matrix values.

In the chapter \textit{Results} we present the results after testing our tools with brain pathways data from \textbf{\textit{Human Connectome Projects}} \cite{CCF}. Also the results were compared with the results from DIPY registration method. In the last chapter \textit{Conclusion} we give advice of how can we improve our tools and further work that can be done.

\end{document}


