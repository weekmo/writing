\documentclass[../structure.tex]{subfiles}
%\usepackage{../mypkg}
\begin{document}
\chapter{Introduction}
\section{Brain Fiber Pathways}
	\paragraph{The human brain} is the central organ of the human nervous system, it is made up of two main components, gray matter and white matter. Scientists have learned a lot about gray and white matter and the two halves of the brain through autopsies and imaging techniques and by studying diseases or conditions associated with brain damage.
	\paragraph{White Matter}refers to areas of the central nervous system that are mainly made up of myelinated axons, also called tracts or fiber pathways \cite{Blumenfeld2010}. Long thought to be passive tissue, white matter affects learning and brain functions, modulating the distribution of action potentials, acting as a relay and coordinating communication between different brain regions \cite{Fields2008}.
	\\White matter is composed of bundles, which connect various gray matter areas of the brain to each other, and carry nerve impulses between neurons. Myelin acts as an insulator, which allows electrical signals to jump, rather than coursing through the axon, increasing the speed of transmission of all nerve signals \cite{Klein2008}.
 
		%\subsection{Anterior Thalamic Radiation (ATR)}
		\paragraph{Anterior Thalamic Radiation (ATR)}
		%The anterior limb of the internal capsule (ALIC) is a white matter structure, the medial portion of which includes the anterior thalamic radiation (ATR) carrying nerve fibers between thalamus and prefrontal cortex.\\
		%The term anterior thalamic radiations
		refers to fiber pathways that connect the anterior nuclear group of the thalamus and the midline nuclear group of the thalamus with the frontal lobe through the anterior thalamic peduncle, the anterior limb of the internal capsule and other parts of the cerebral white matter \cite{Washington1994}\cite{Grimm2018}.\\
		ATR abnormalities have a possible link with cognitive abnormalities and negative symptoms in schizophrenia\cite{Mamah2010}.
		
		\paragraph{Corpus Callosum (CC)} is a wide, flat bundle of nerve fibers, located at the longitudinal fissure beneath the cortex, which acts a link between the two hemispheres of the brain and facilitates communication between them. The term corpus callosum means 'tough body' in Latin. With approximately 200 - 250 million contralateral axonal projections to its credit, it is the largest among the various white matter structures in the central nervous system.
The anterior portion of this structure is called '\textbf{genu}', while the posterior structure is called '\textbf{splenium}'. In between its anterior and posterior portions, lies the 'truncus' or its '\textbf{body}'. Studies have revealed that the anterior of corpus callosum in left-handed people is 11 percent larger than that of right-handed people \cite{PDD2015}.\\\
		\paragraph{Genu of the Corpus Callosum (genu)}
		 refers to the rostral most portion of the corpus callosum. It is bounded caudally by the body of the corpus callosum and ventrocaudally by the rostrum of the corpus callosum \cite{Washington1994}.\\
		\paragraph{Splenium of the Corpus Callosum (splenium)}
		 refers to the caudal most portion of the corpus callosum. It is bounded rostrally by the body of the corpus callosum \cite{Washington1994}.\\
%It overlaps the tela chorioidea of the third ventricle and the mid-brain, and ends in a thick, convex, free border. A sagittal section of the splenium shows that the posterior end of the corpus callosum is acutely bent forward, the upper and lower parts being applied to each other \cite{PDD2015}.\\
\paragraph{Body of Corpus Callosum (truncus)} 
		 refers to the portion of the corpus callosum located between the genu of the corpus callosum and the splenium of the corpus callosum. In a common parcellation, corpus callosum, it is divided into four parts: the rostral body of the corpus callosum, the anterior midbody of the corpus callosum, the posterior midbody of the corpus callosum and the isthmus of the corpus callosum \cite{Washington1994}.
		\paragraph{Cingulum (Cing)}
		refers to a fiber pathway that runs longitudinally in the cingulate white matter; it connects portions of the cingulate gyrus, the parietal lobe and the prefrontal cortex with the parahippocampal gyrus and adjacent structures of the temporal lobe. "All connectios entering and exiting the cingulate gyrus pass through the cingulum bundle". It is composed of the Cingulum ammonale and the Cingulum limitans \cite{Washington1994}.\\
		%In neuroanatomy, the cingulum is a collection of white matter fibers projecting from the cingulate gyrus to the entorhinal cortex in the brain, allowing for communication between components of the limbic system. It forms the white matter core of the cingulate gyrus, following it from the subcallosal gyrus of the frontal lobe beneath the rostrum of corpus callosum to the parahippocampal gyrus and uncus of the temporal lobe.
Cingulum receives afferent fibers from the parts of the thalamus that are associated with the spinothalamic tract. This, in addition to the fact that the cingulum is a central structure in learning to correct mistakes, indicates that the cingulum is involved in appraisal of pain and reinforcement of behavior that reduces it \cite{Brodal2016}.
%Cingulotomy, the surgical severing of the anterior cingulum, is a form of psychosurgery used to treat depression and OCD.
%The cingulum was one of the earliest identified brain structures.\\
%Anatomy and Function The cingulum is described from various brain images as a C shaped structure within the brain that wraps around the frontal lobe to the temporal lobe right above the corpus callosum. It is located beneath the cingulate gyrus within the medial surface of the brain therefore encircling the entire brain. There are two primary parts of the cingulate cortex, as is typical with most brain structures. There is the posterior cingulate and anterior cingulate. The anterior is linked to emotion, especially apathy and depression. Here function and structure changes are related meaning any change within this structure would lead to a function change, particularly behavioral because of its function involving emotions. Damage to this area can have various effects on mental disorders and mental health. The posterior section is more related to cognitive functions. This can include attention, visual and spatial skills, working memory and general memory. Because of its location, the cingulum is very important to brain structure connectivity and the integration of information that it receives \cite{JaredTanner2010}.
	
	%Editing stopped here last time	
	
		\paragraph{Corticospinal Tract (CST)}
		refers to a fiber pathway from the cerebral cortex to the spinal cord. Its fibers originate from pyramidal neurons of the precentral gyrus, and on their way to the spinal cord, they pass through parts of the cerebral white matter (including the posterior limb of the internal capsule), the crus cerebri, the longitudinal pontine fibers, the pyramid of the medulla (where they are known as the pyramidal tract) and the pyramidal decussation. In the ducussation, some fibers cross to the other side of the brainstem to form the lateral corticospinal tract. Those fibers that do not cross split to form the anterolateral corticospinal tract and the anterior corticospinal tract \cite{Washington1994}.
		\paragraph{Fornix (Fornix)}
		The fornix (Latin, "vault" or "arch") is a C-shaped bundle of fibers (axons) in the brain, and carries signals from the hippocampus to the hypothalamus.
The fibres begin in the hippocampus on each side of the brain (where they are also known as the fimbria); the separate left and right sides are each called the crus of the fornix. The bundles of fibres come together in the midline of the brain, forming the body of the fornix. The inferior edge of the septum pellucidum (a membrane that separates the two lateral ventricles) is attached to the upper face of the fornix body.
%The body of the fornix travels anteriorly and divides again near the anterior commissure. The left and right parts separate, but there is also an anterior/posterior divergence.
%The posterior fibres (called the postcommissural fornix) of each side continue through the hypothalamus to the mammillary bodies; then to the anterior nuclei of thalamus, which project to the cingulate cortex.
%The anterior fibers (precommissural fornix) end at the septal nuclei and nucleus accumbens of each half of the brain \cite{PDD2015}.\\
		While its exact function and importance in the physiology of the brain are still not entirely clear, it has been demonstrated that surgical transection – the cutting of the fornix along its body – can cause memory loss \cite{HenryGray1918}. %There is some debate over what type of memory is affected by this damage, but it has been found to most closely correlate with recall memory rather than recognition memory. This means that damage to the fornix can cause difficulty in recalling long-term information such as details of past events, but it has little effect on the ability to recognize objects or familiar situations \cite{HenryGray1918}.
		
		\paragraph{Inferior Fronto-occipital Fasciculus (IFO)}
		The occipitofrontal fasciculus passes backward from the frontal lobe, along the lateral border of the caudate nucleus, and on the medial aspect of the corona radiata; its fibers radiate in a fan-like manner and pass into the occipital and temporal lobes lateral to the posterior and inferior cornua \cite{PDD2015}.
		\paragraph{Inferior Longitudinal Fasciculus (ILF)}
		The inferior longitudinal fasciculus connects the temporal lobe and occipital lobe, running along the lateral walls of the inferior and posterior cornua of the lateral ventricle.
The existence of this fasciculus independent from the occipitotemporal fasciculus has been questioned for the human being, such that it has been proposed that the term inferior longitudinal fasciculus be replaced by the term "occipitotemporal projection" \cite{PDD2015}.
		\paragraph{Superior Longitudinal Fasciculus (SLF)}
		%refers to a fiber pathway in the cerebral white matter. It is composed of fibers that connect the cortex of the frontal lobe with cortex of the occipital lobe and the temporal lobe. Some authors refer to the connection with the temporal lobe as the arcuate fasciculus \cite{Washington1994}.\\
		is a pair of long bi-directional bundles of neurons connecting the front and the back of the cerebrum. Each association fiber bundle is lateral to the centrum ovale of a cerebral hemisphere and connects the frontal, occipital, parietal, and temporal lobes. The neurons pass from the frontal lobe through the operculum to the posterior end of the lateral sulcus where numerous neurons radiate into the occipital lobe and other neurons turn downward and forward around the putamen and radiate to anterior portions of the temporal lobe \cite{PDD2015}.
		
		\paragraph{Ventral Tegmental Area (VTA)}
		is in the midbrain, situated adjacent to the substantia nigra. Although it contains several different types of neurons, it is primarily characterized by its dopaminergic neurons, which project from the VTA throughout the brain.
		The VTA is considered an integral part of a network of structures, together known as the reward system, that are involved in reinforcing behavior.
\subsection{From Thesis}
	The surface registration of two or more three-dimensional data sets by definition
is the process of transforming the data and bringing it together in
order for the semantically corresponding components in the data sets to be
aligned. The scanning of three-dimensional object often gives occasion for the application of surface registration, since perfect conditions are rarely the case as the data sets are produced by scanning objects from dierent view
points and assigning them in dierent coordinate systems. Those scans produce
noisy and incomplete surfaces that should later be combined and the
missing data should be reconstructed, for which surface registration is used,
thus spreading its importance in a variety of fields as computer graphics and
computer vision.
Surface registration can be divided in two types: rigid and nonrigid, depending
on the types of transformations that need to be done on the data in
order for a proper alignment to be achieved. In contrast to the rigid surface
registration, the nonrigid allows not only translations, rotations and reflections
to be used, but also deformations, which leads to much more possible
registration results that can be considered.
This master thesis will examine that case of nonrigid surface registration
to both mesh and point cloud data. Those two types of problems can be
reduced to one when translating both types to a connected undirected graph,
using the triangulation of the mesh for the edges and a KNN algorithm for
constructing a graph from the point cloud. The thesis is focused on an already
existing algorithm, developed and described in [1] by Brian Amberg, Sami
Romdhani and Thomas Vetter. It works to find a registration of a template
surface over a target one (Figure 1.1), using an iterative closest point method
and imposing constraints on the deformation that is occurring on each step,
defining a cost function that needs to be minimized by solving an equation.
The iterations of the algorithm are guided by a changing stiness parameter
that controls the nature of the transformations that are applied, starting
from more global ones and reducing to local ones by the end of the run. The
goal of the master thesis is to propose an optimization of the given algorithm,
alternating both its steps and cost function, adapting it in a way so that it
works much faster for meshes and point clouds with a large number of points.
Described as patch-based, the optimization works by clustering the template
graph and performing deformations on the patches instead on the vertices,
which means that clusters are moving together, decreasing the computation
time of solving the equation due to fewer number of elements being
considered. The vertices get transformed depending on the way the cluster
they are in is moving, as well as its neighboring clusters, which is achieved
using soft membership weights. The algorithm starts with a big number of
patches that get divided into smaller ones on each step, this way the new
algorithm mimics the old one by applying global deformations at first and
then moving the vertices locally to the target. The proposed algorithm is
a nonrigid optimal step ICP algorithm as the one in [1], with a dierent
convergence criterion.
The idea of the proposed method went through multiple iterations itself
before reaching the final definition that is depicted in this text. Dierent
method were tested for the clustering until the best one was found and applied.
The role of the clustering also grew bigger when the initial idea, to
apply the original algorithm on a one-time clustered template, proved unsatisfactory
in practice.
The thesis consists of six Chapters: Introduction, Related Works, Method,
Implementation, Results and the Conclusion. In the next Chapter Related
Works the state of the art is reviewed. The main theoretical Chapter Method
starts with briefly defining the setup of the algorithm, describing the preliminary
steps and the essence of the optimization - the clustering. After this the
original cost function from [1] is reviewed and its alternations are explained,
finishing with the whole algorithm idea step by step. In the Chapter Implementation
we talk briefly about how the implemented tool works, what
libraries are used for it and how the output looks. The Chapter Results is
where the experiments are described and both algorithms, the original one
from [1] and the optimized one from this thesis, are compared in terms of
time and registration. The thesis ends with a Conclusion Chapter, where
everything is summarized.
	\section{Registration}
	The 3D-3D registration problem (also known as 3D-3D alignment, 3D absolute orientation, 3D pose ) is one of the outstanding and very basic problems in computer vision. In this problem, two sets of 3D points are given and the task is to optimally align these two sets of points by estimating a best transformation between them. Due to its fundamental importance, it arises as a subtask in many different applications (e.g., object recognition, tracking, range data fusion, graphics, medical image alignment, robotics and structural bioinformatics etc \cite{Li2007}.
		\subsection{Iterative closest point (ICP)}
		ICP is an algorithm employed to minimize the distance between two clouds of points. %\cite{Besl1992}\cite{Chen1992}\cite{Zhang1994}.
		In ICP (in our case) one point cloud (vertex cloud), the reference, or target, is kept fixed, while the other one, the source, is transformed to best match the reference. The algorithm iteratively revises the transformation (combination of translation and rotation) needed to minimize a distance from the source to the reference point cloud. ICP is one of the widely used algorithms in aligning three dimensional models given an initial guess of the rigid body transformation required \cite{Rusinkiewicza2001}.
		\subsection{PCA Transformation}
		\subsection{Clustering}
		\subsection{Distance Functions}
			\paragraph{KD Tree}
			\paragraph{Point Cloud based}
			\paragraph{Fiber Based}

\end{document}
