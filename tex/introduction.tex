\documentclass[../structure.tex]{subfiles}
%\usepackage{../mypkg}
\begin{document}
\chapter{Introduction}
In order to compare the differences in the fiber pathways of cases and control to study various brain related disorders, this thesis focuses on customizing \textit{Non-rigid Iterative Closest Point Registration}, which is discussed in \cite{Amberg2007}, to process 3D brain fiber pathways. Due to differences between pathways in different subjects , the alignment of this type of data is not a trivial task. Also, there is variability in the shapes of the same fiber pathways in different brains and alignment between the left and right side of the fiber pathway can also be challenging. We illustrate the method in details in chapter \textit{Methods and Implementation} and we develop tools that can contribute of solving the difficulty of dealing with such data. The tools can assign local transformation which gives sufficient registration of fiber pathways. Dipy library and one linear method used an optimizer to reach the optimal combination of affine matrix values.
LSQR, a linear solver is the state of art method which is used to reduce the cost function or euclidean distance in our case to reach the minimal overall distance with respect to stiffness which prevents the template graph to totally map to the target graph.
An advantage of using LSQR is linearly solving the cost function which consists of matrices by assigning local transformation (local transformation means affine matrix for each vertex in the template graph). But the method is slow and can improved in performance by applying hierarchical clustering (already mentioned in conclusion).

In chapter \textit{Result} we present result after testing our tools with real data from patients and controls which is collected at \textit{UKB Universitätsklinikum Bonn}. In the last chapter \textit{Conclusion} we give advice of how can we improve our tools and further work that can be done.

\end{document}


