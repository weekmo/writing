\documentclass[../structure.tex]{subfiles}
\begin{document}
\chapter{Results}
\hspace{2em}After implementing non-rigid ICP, we test the tools we have developed to assess the algorithm with data collected from \textbf{\textit{Human Connectome Projects}} \cite{CCF}. The number of samples used for testing is four with eleven pathways (bundles) in each side of the brain.

% Statistic
\begin{center}
\begin{table}[h!]
	\begin{tabular}{| c | c  c  c | c  c  c |}
	%\begin{tabular}{*7c}
	\toprule
	&\multicolumn{3}{c}{Tracts}&\multicolumn{3}{c}{Points}\\
Pathways&Average&Max&Min&Average&Max&Min\\
\midrule
%\hline
ATR&266.75&603&106&26443.875&69283&9623\\
rostrum&1853.375&2020&1633&162936.625&188143&137536\\
Cing&1103.375&1369&872&167488.25&244088&100762\\
CST&1649.25&2044&1476&235265.375&308769&170535\\
Fornix&399&497&308&26618&37142&16703\\
genu&2134.5&2333&1967&166567.625&236384&144492\\
IFOF&862.875&985&774&128850.875&162086&90424\\
ILF&3751&4392&2950&566332.25&673060&367155\\
SLF&1333.125&1633&1044&174943.5&251994&105245\\
splenium&2209.625&2335&2063&206526.875&244341&182993\\
VTA&327.625&679&150&27784.125&54904&14982\\
\bottomrule
	\end{tabular}
\caption{The statistics of tracts (streamlines) and points in data}
\label{table:data}
\end{table}
\end{center}

\section{DIPY registration Method}
We compare our tools with a registration method discussed in \cite{Garyfallidis2015} and implemented in \textit{DIPY} package.

The method starts by reducing the number of points in each tract to a certain number of point, which is necessary for the method to have the same number of points in each tract.

DIPY method gets the initial orientation by placing both bundles (streamlines) in the origin point of the 3D Euclidean space. In this case, it considers only translation but not other transformations. To solve this problem and to be fare on the comparison, we use PCA tool that we develop to get the initial alignment before applying the DIPY method.

DIPY calculate the distance between template bundle and target bundle using a function called \textit{minimum average direct-flip distance (MDF)}. MDF is asymmetric
distance function, which can be applied only when all tracts have the same number of points, that is one reason why we reduce the number of point before applying DIPY registration, another reason is increasing the speed of distance calculation \cite{Garyfallidis2012}.

The method then uses optimizer to select the best combination of variables, which later converted to affine matrix, while reducing the overall distance between template bundle.

\begin{comment}
\section{ICP testing steps}
\begin{itemize}
\item Read bundles from \textit{ply} files
\item Save the visual orientation of the original alignment
\item Apply PCA
\item Visualize bundles after PCA and compare with original visual orientation
\item If the visual inspection was positive and PCA improved the alignment, we consider its result, otherwise we just flip the template bundle or consider the original orientation
\item Generate distances histogram between two bundles to select the distance threshold
\item Start the registration and iterate until there is no more improvement on the alignment
\end{itemize}
\end{comment}

\section{Experiment (1)}
\hspace{2em}In this experiment we use both side of Thalamic Radiation (ATR) fiber pathway, whereas the template bundle is the ATR on the right side and the target bundle is the left side of the ATR from the same patient.

\begin{figure}[h!]
\centering
\includegraphics[scale=.8]{101_img_original}
\captionsetup{justification=centering}
\caption{The original orientation of ATR left side (red) as target bundle and right side of ATR (blue) as template bundle}
\label{fig:img_original}
\end{figure}

\begin{figure}[h!]
\centering
\includegraphics[scale=.3]{101_img_PCA}
\captionsetup{justification=centering}
\caption{The Orientation of ATR left side (red) as target bundle and right side of ATR (blue) as template bundle after applying PCA}
\label{fig:img_PCA}
\end{figure}
\pagebreak

First, we read the data (left and right side of ATR) using a tool we customized from \textit{PlyFile} package to suit our data. Then we apply PCA and visually inspect the result as shown in Figure \ref{fig:img_original} \ref{fig:img_PCA}, if PCA improve the alignment, we consider it, otherwise we accept the original orientation or just flip the template bundle if they are from different side of the brain.

As we can see in this experiment that PCA improved the alignment, so we go ahead and generate distances histogram as shown in Figure \ref{fig:hist_PCA} to select the threshold, which uses to build weights matrix.

We also show the original distances histogram to show the improvement of the distances as well in Figure \ref{fig:hist_original}.

\begin{figure}[h!]
\centering
\includegraphics[scale=.9]{101_hist_original}
\captionsetup{justification=centering}
\caption{The original distances histogram}
\label{fig:hist_original}
\end{figure}

\begin{figure}[h!]
\centering
\includegraphics[scale=1]{101_hist_PCA}
\captionsetup{justification=centering}
\caption{The distances histogram after PCA}
\label{fig:hist_PCA}
\end{figure}

% Stoped here last time

concatenate each bundle separately to have point cloud for each.

Then we visualize two graphs and plot the distances between correspondences points in histograms before and after applying PCA. Depending on the result we have and whether PCA improved the alignment or not, we decide whether to use PCA or not; if PCA improves the alignment, then it is used, otherwise we skip PCA. The Figures \ref{fig:hist_original} \ref{fig:img_original} \ref{fig:hist_PCA} \ref{fig:img_PCA} below show visual orientations and histograms before and after PCA.




\begin{figure}[h!]
\centering
\includegraphics[scale=.3]{101_img_PCA}
\captionsetup{justification=centering}
\caption{The Orientation after PCA}
\label{fig:img_PCA}
\end{figure}
\vspace{4em}

\begin{figure}[h!]
\centering
\includegraphics[scale=.6]{101_img_original}
\captionsetup{justification=centering}
\caption{The original orientation}
\label{fig:img_original}
\end{figure}


\pagebreak
\vspace{4em}
\begin{figure}[h!]
\centering
\includegraphics[scale=1]{101_hist_PCA}
\captionsetup{justification=centering}
\caption{The distances histogram after PCA}
\label{fig:hist_PCA}
\end{figure}

\begin{figure}[h!]
\centering
\includegraphics[scale=.3]{101_img_PCA}
\captionsetup{justification=centering}
\caption{The Orientation after PCA}
\label{fig:img_PCA}
\end{figure}
\vspace{4em}
\pagebreak
%\vspace{2em}
When we decide whether to use PCA or not, we select the threshold depending on the histogram. In this case, as we decide to use PCA, we select a threshold equal to seven and around 98\% of points will be included as correspondences in the first iteration. In other words, 98\% of the weight matrix $W$ will be \textit{ones}, as explained in chapter \textit{Methods and implementation}. Then we run \textit{LSQR} to solve the cost function iteratively until the distance between the last iteration and the current iteration reach a selected value. The value of \textit{acond}, which presents how good conditioned LSQR, returned by \textit{scipy.sparse.linalg.lsqr} is 51810683.6. The results are shown in Figures \ref{fig:hist_ICP} \ref{fig:img_ICP} below. Later we compare our results with non-linear method called \textit{Dipy} 

\begin{figure}[h!]
\centering
\includegraphics[scale=1]{101_hist_ICP}
\captionsetup{justification=centering}
\caption{The distances histogram after ICP}
\label{fig:hist_ICP}
\end{figure}

\begin{figure}[h!]
\centering
\includegraphics[scale=.4]{101_img_ICP}
\captionsetup{justification=centering}
\caption{The Orientation after ICP}
\label{fig:img_ICP}
\end{figure}
\pagebreak
We align the same bundles with the method discussed in \cite{Garyfallidis2012} and implemented in \textit{DIPY dipy.align.streamlinear.StreamlineLinearRegistration}. The method starts by reducing the number of points in each tract to a certain number of point, which is necessary for the method to have the same number of points in each tract, and applies \textit{minimum average direct-flip distance (MDF)} to calculate the distance. The method then uses optimizer to select the best combination of variables, which later converted to affine matrix, to reduce the overall distance between template bundle and the target bundle. The method gets the initial orientation by bringing both graphs to the origin point of the 3D euclidean space, but it does not take care of the initial alignment. Therefore, we use the PCA tool that we develop to get the initial alignment and produce the results shown in Figures \ref{fig:dipy_hist} \ref{fig:img_dipy}

\begin{figure}[h!]
\centering
\includegraphics[scale=1]{101_dipy_hist}
\captionsetup{justification=centering}
\caption{The distances histogram after DIPY}
\label{fig:dipy_hist}
\end{figure}

\begin{figure}[h!]
\centering
\includegraphics[scale=.3]{101_img_dipy}
\captionsetup{justification=centering}
\caption{The Orientation after DIPY}
\label{fig:img_dipy}
\end{figure}
\pagebreak

The second test is done for the right side of the ATR by transforming it locally and globally. Essentially, what we mean by local transformation is that we apply different affine transformation matrices for each point in the bundle and local means one affine transformation matrix for all points (no deformation). Then we try to align it again with the original one. We show the results in Figures \ref{fig:hist_original_def} \ref{fig:img_original_def} of local transformation only because it is relatively straightforward for global transformation. The condition value of LSQR we got is 5741672.

\begin{figure}[h!]
\centering
\includegraphics[scale=1]{102_hist_original}
\captionsetup{justification=centering}
\caption{The distances histogram after deforming a bundle and measuring the distances from the original one}
\label{fig:hist_original_def}
\end{figure}

\begin{figure}[h!]
\centering
\includegraphics[scale=.3]{102_img_original}
\captionsetup{justification=centering}
\caption{The orientation after deformation}
\label{fig:img_original_def}
\end{figure}
\pagebreak

Next, we align them with DIPY to get the results in Figures \ref{fig:hist_dipy_def} \ref{fig:img_dipy_def}.

\begin{figure}[h!]
\centering
\includegraphics[scale=1]{102_hist_dipy}
\captionsetup{justification=centering}
\caption{The distances histogram after deforming a bundle and aligning it with the original one using DIPY}
\label{fig:hist_dipy_def}
\end{figure}

\begin{figure}[h!]
\centering
\includegraphics[scale=.3]{102_img_dipy}
\captionsetup{justification=centering}
\caption{The orientation after deforming a bundle and aligning it with the original one using DIPY}
\label{fig:img_dipy_def}
\end{figure}
\pagebreak
We then apply non-rigid ICP registration to get the results in Figures \ref{fig:hist_icp_def} \ref{fig:img_icp_def}.

\begin{figure}[h!]
\centering
\includegraphics[scale=1]{102_hist_ICP}
\captionsetup{justification=centering}
\caption{The distances histogram after deforming a bundle and aligning it with the original one using ICP}
\label{fig:hist_icp_def}
\end{figure}

\begin{figure}[h!]
\centering
\includegraphics[scale=.3]{102_img_ICP}
\captionsetup{justification=centering}
\caption{The orientation after deforming a bundle and aligning it with the original one using ICP}
\label{fig:img_icp_def}
\end{figure}

\pagebreak

Applying PCA sometimes does not find the right orientation of bundles due to similarity of vertices distribution and that gives similar principle components as shown below:


	\begin{figure}[h!]
	\centering
	\includegraphics[scale=0.5]{02_img_original}
	\captionsetup{justification=centering}
	\caption{The original coordinate before PCA}
	\label{fig:all_brain}
	\end{figure}
	
	\begin{figure}[h!]
	\centering
	\includegraphics[scale=0.3]{02_img_PCA}
	\captionsetup{justification=centering}
	\caption{The coordinate when PCA fails to find the best alignment}
	\label{fig:all_brain}
	\end{figure}
\pagebreak

Finally, we found that our tool can handle the deformation easily (elastic registration) but \textit{DIPY} can only handle the size by scaling the template bundles.

The \textit{DIPY} method is much faster than our method because it is implemented to reduce the number of points used to calculate the distance. In our test, each tract reduced to 20 points rather than the real number of points as shown in table \{\ref{table:data}\}. Furthermore, as mentioned in \cite{Garyfallidis2012}, \textit{DIPY} uses a minimum average direct-flip distance (MDF) which considers the tract to tract distance whereas our tool considers the point to point distance. To improve our tool, hierarchical clustering bundles can be done and soft membership alignment can be applied.

\end{document}


