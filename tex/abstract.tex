\documentclass[../structure.tex]{subfiles}
%\usepackage{../mypkg}
\begin{document}
\chapter*{Abstract}
\hspace{.5cm} The registration problem (also known as alignment, absolute orientation) is one of the outstanding and very basic problems in computer vision. In this problem, two or more datsets of points are given and the task is to optimally align them by estimating a best transformation (combination of translation, rotation and scaling). Due to its fundamental importance, it arises as a subtask in many different applications (e.g., object recognition, tracking, range data fusion, graphics, medical image alignment, robotics and structural bioinformatics etc \cite{Li2007}.
\\In This thesis we show how to extend the ICP framework to nonrigid registration, while retaining the convergence properties of the original algorithm. The resulting optimal step nonrigid ICP framework allows the use of different regularisations, as long as they have an adjustable stiffness parameter. The registration loops over a series of decreasing stiffness weights, and incrementally deforms the template towards the target, recovering the whole range of global and local deformations. To find the optimal deformation for a given stiffness, optimal iterative closest point steps are used. Preliminary correspondences are estimated by a nearestpoint search. Then the optimal deformation of the template for these fixed correspondences and the active stiffness is calculated. Afterwards the process continues with new correspondences found by searching from the displaced template vertices. We present an algorithm using a locally affine regularisation which assigns an affine transformation to each vertex and minimises the difference in the transformation of neighbouring vertices. It is shown that for this regularisation the optimal deformation for fixed correspondences and fixed stiffness can be determined exactly and efficiently. The method succeeds for a wide range of initial conditions, and handles missing data robustly. It is compared qualitatively and quantitatively to other algorithms using synthetic examples and real world data\cite{Amberg2007}.
\end{document}