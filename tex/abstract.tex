\documentclass[../structure.tex]{subfiles}
%\usepackage{../mypkg}
\begin{document}
\chapter{Abstract}
\hspace{2em}The registration problem or sometimes called alignment or absolute orientation is one of fundamental problem in computer vision. If we have two objects and we need to align them that means we should reduce the distance between them by making one object fix and move the other one to the closest distance, this simple form of alignment called rigid transformation because it doesn't change the shape of the template object, where as the non-rigid transformation is when deformation occur to the shape of the template object. Due to fundamental importance of registration in computer vision, it is necessary step in many different applications, for instance: object recognition, tracking, range data fusion, graphics, medical image alignment, robotics and structural bioinformatics etc \cite{Li2007}.

\hspace{2em}As we mention, the registration is important in medical image alignment, we are going to apply it on Brain fiber pathways, which is 3D images data. This type of data is not easy to align due to slightly different shape of the same fiber pathway in different brains, and sometimes alignment occur between left side and right side of the fiber pathway.

\hspace{2em}One of the famous methods to solve this problem is Iterative Closest Point (ICP), which we apply in this thesis. The main idea is in each point on the moving object we find the closest point on the fixed object, with respect to threshold, and try to minimize the overall distance between them, the minimal overall distance can be reached by solving the cost function that represent the distance and stiffness, which we will discuss later in this thesis. Iterative Closest Point (ICP) can be applied by different algorithms, in our case we use Least squares (LSQR) algorithm to solve our problem and we have sufficient results.
\end{document}