\documentclass[../structure.tex]{subfiles}
%\usepackage{../mypkg}
\begin{document}
\chapter{Abstract}
The registration problem, sometimes called alignment or absolute orientation, is one of the fundamental problems in computer vision. If we have two objects that we need to align, this would require that we reduce the distance between them by fixing one object in its place and moving the other one (i.e., template object) to the closest distance to the first. This simple form of alignment is called rigid transformation because it does not change the shape of the template object, whereas a non-rigid transformation is when a deformation of the shape of the template object occurs. Registration is of fundamental importance in computer vision, thus, it is a necessary step in many different applications. These include object recognition, tracking, range data fusion, graphics, robotics and structural bioinformatics \cite{Li2007}.

Registration is also applied in and of crucial importance to medical image alignment. Here we apply it on brain fiber pathways, which is data of 3D images. The alignment of this type of data is not a trivial task as there is variability  in the shapes of the same fiber pathways in different brains and alignment between the left and right side of the fiber pathway can also be challenging.

One of the more notable methods to solve this problem is Iterative Closest Point , which we apply in this thesis. The principal idea is that in each point on the moving object, we find the closest point on the fixed object, which is less than a calculated threshold distance, and try to minimize the overall distance between these two objects. The minimal overall distance can be determined by solving the cost function, using the Least Squares (LSQR) algorithm iteratively until the distance between the last iteration and the current iteration is less than a parameter selected by the user. In this work we develop tools that implements these methods to handle local alignment and address the aforementioned difficulties associated with 3D images of fiber pathways (e.g., variations in the shapes of fiber tracts, aligning fiber pathways from both sides of the brain). Our results demonstrate the effectiveness of this approach to achieve these goals.
\end{document}
