\documentclass[../structure.tex]{subfiles}
%\usepackage{../mypkg}
\begin{document}
\chapter{Introduction}
\section{Brain Structure (Fiber Pathways)}
	\paragraph{Human brain}, which is our focus in this thesis, is the central organ of the human nervous system, it is made up of two main components, gray matter and white matter. Scientists have learned a lot about gray and white matter and the two halves of the brain through autopsies and imaging techniques and by studying diseases or conditions associated with brain damage.
	\\The thesis focused on \textbf{White Matter} which is refers to areas of the central nervous system that are mainly made up of myelinated axons, also called tracts or fiber pathways \cite{Blumenfeld2010}. It is composed of bundles, which connect various gray matter areas of the brain to each other, and carry nerve impulses between neurons. Myelin acts as an insulator, which allows electrical signals to jump, rather than coursing through the axon, increasing the speed of transmission of all nerve signals \cite{Klein2008}.
	\\Long thought to be passive tissue, white matter affects learning and brain functions, modulating the distribution of action potentials, acting as a relay and coordinating communication between different brain regions \cite{Fields2008}.
	
	The Human Brain consits of these tracts in left and right side:
    \begin{itemize}
        \item Anterior Thalamic Radiation (ATR)
        \item Corpus Callosum (CC)
        \item Genu of the Corpus Callosum (genu)
        \item Splenium of the Corpus Callosum (splenium)
        \item Body of Corpus Callosum (truncus)
        \item Body of Corpus Callosum (truncus)
		\item Cingulum (Cing)
		\item Corticospinal Tract (CST)
		\item Inferior Fronto-occipital Fasciculus (IFO)
		\item Inferior Longitudinal Fasciculus (ILF)
		\item Superior Longitudinal Fasciculus (SLF)
		\item Ventral Tegmental Area (VTA)
	\end{itemize}
\section{Registration}
	The registration problem (also known as alignment, absolute orientation) is one of the outstanding and very basic problems in computer vision. In this problem, two or more datsets of points are given and the task is to optimally align them by estimating a best transformation (combination of translation, rotation and scaling). Due to its fundamental importance, it arises as a subtask in many different applications (e.g., object recognition, tracking, range data fusion, graphics, medical image alignment, robotics and structural bioinformatics ... etc) \cite{Li2007}.
		\subsection{Iterative closest point (ICP)}
		 ICP, which is an algorithm employed to minimize the distance between two or more points clouds, is one of the widely used algorithms in aligning three dimensional models given an initial guess of the rigid body transformation required \cite{Zhang1994}.
		 In ICP (in our case) one points cloud (vertex cloud), the reference, or target, is kept fixed, while the other one, the source, is transformed to best match the reference. The algorithm iteratively revises the transformation (combination of translation, rotation and scaling) needed to minimize a distance from the source to the reference points cloud.
\section{PCA Transformation}
PCA is mathematically defined as an orthogonal linear transformation that transforms the data to a new coordinate system such that the greatest variance by some projection of the data comes to lie on the first coordinate (called the first principal component), the second greatest variance on the second coordinate, and so on\cite{Jolliffe2002}.
PCA is used in the code as a preliminary step, so that the template and target are aligned as much as possible before the registration can begin.
\section{Cost Function}
Cost function or loss function maps values of one or more variables onto a real number intuitively representing some "cost" associated with the event. An optimization problem seeks to minimize a loss function\cite{Wald1950}. In our case we use total distance as cost, to obtain the distances KDTree is used.
\subsection{K-D Tree}
The k-d tree is a binary tree in which every leaf node is a k-dimensional point. Every non-leaf node can be thought of as implicitly generating a splitting hyperplane that divides the space into two parts, known as half-spaces. Points to the left of this hyperplane are represented by the left subtree of that node and points to the right of the hyperplane are represented by the right subtree. The hyperplane direction is chosen in the following way: every node in the tree is associated with one of the k dimensions, with the hyperplane perpendicular to that dimension's axis \cite{JonLouisBentley1975}.
%So, for example, if for a particular split the "x" axis is chosen, all points in the subtree with a smaller "x" value than the node will appear in the left subtree and all points with larger "x" value will be in the right subtree. In such a case, the hyperplane would be set by the x-value of the point, and its normal would be the unit x-axis\cite{JonLouisBentley1975}.
\section{Combination}
PCA is used in the code as a preliminary step, so that the template and target are aligned as much as possible before the registration can begin.


We show how to extend the ICP framework to nonrigid registration, while retaining the convergence properties of the original algorithm. The resulting optimal step nonrigid ICP framework allows the use of different regularisations, as long as they have an adjustable stiffness parameter. The registration loops over a series of decreasing stiffness weights, and incrementally deforms the template towards the target, recovering the whole range of global and local deformations. To find the optimal deformation for a given stiffness, optimal iterative closest point steps are used. Preliminary correspondences are estimated by a nearestpoint search. Then the optimal deformation of the template for these fixed correspondences and the active stiffness is calculated. Afterwards the process continues with new correspondences found by searching from the displaced template vertices. We present an algorithm using a locally affine regularisation which assigns an affine transformation to each vertex and minimises the difference in the transformation of neighbouring vertices. It is shown that for this regularisation the optimal deformation for fixed correspondences and fixed stiffness can be determined exactly and efficiently. The method succeeds for a wide range of initial conditions, and handles missing data robustly. It is compared qualitatively and quantitatively to other algorithms using synthetic examples and real world data \cite{Amberg2007}.
\end{document}
