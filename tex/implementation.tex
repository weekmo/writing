\documentclass[../structure.tex]{subfiles}
\begin{document}

\chapter{implementation}

\end{document}